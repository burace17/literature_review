\documentclass{article}
\usepackage[american]{babel}
\usepackage[utf8]{inputenc}
\usepackage[margin=1in]{geometry}
\usepackage[citestyle=apa,style=apa,backend=biber]{biblatex}
\DeclareLanguageMapping{american}{american-apa}
\addbibresource{bibliography.bib}
\usepackage{outlines}
\usepackage{enumitem}
\setenumerate[1]{label=\Roman*.}
\setenumerate[2]{label=\Alph*.}
\setenumerate[3]{label=\roman*.}
\setenumerate[4]{label=\alph*.}

\begin{document}
\begin{outline}[enumerate]
\1 Introduction
	\2 Internet of Things (IoT) security is currently not adequate.
		\3 Many devices are vulnerable to various types of attacks.
		\3 In September 2016, a security researcher had his website attacked by thousands of compromised IoT devices. (\cite{Krebs}; \cite{Incapsula}).
		\3 A backdoor in Sony IP Cameras was also found in December of 2016 which put 4,250 devices at risk of being compromised by attackers. (\cite{sec}; \cite{Krebs2}).
	\2 Background
		\3 The literature outlines security risks in all layers of the IoT architecture. (\cite{Xiaohui6643029}; \cite{Zhao6746513}; \cite{Suo6188257}).
			\4 However, there is some debate as to which layers need the most attention for future research. (\cite{Zhao6746513}; \cite{Kozlov}).
		\3 The IoT needs a complete stack of protocols to ensure security. 
			\4 The Datagram Transport Layer Security (DTLS) protocol has received a lot of attention from researchers. (\cite{Garcia-Morchon:2013:SII:2462096.2462117}).
			\4 However, other literature suggests that the Host Identity Protocol (HIP) would be a better choice. (\cite{Garcia-Morchon:2013:SII:2462096.2462117}; \cite{Hummen}).
		\3 Some governments are considering action on IoT security
			\4 The Federal Trade Commission released a report detailing their recommendations. (\cite{ftc}).
			\4 European Union lawmakers are considering new rules for manufacturers of IoT devices. (\cite{Stupp}).
			
\1 Methods
\1 Security Risks for IoT Devices
	\2 Current Security Risks
		\3 Perception Layer
			\4 In the perception layer, it is possible for attackers to physically replace components that compromise security. (\cite{Xiaohui6643029}).
			\4 The literature also points out that devices at the perception layer are not powerful enough for complex security protocols. (\cite{Xiaohui6643029}).
			\4 Perception layer devices could also be at risk of attack via traditional methods, like distributed denial of service. (\cite{Zhao6746513}).
		\3 Network Layer
			\4 Network layer, for the most part, has a complete security implementation. (\cite{Zhao6746513}).
			\4 Existing problems include eavesdropping, illegal access, and others. (\cite{Zhao6746513}).
		\3 Application Layer
			\4 At the application layer, traditional security risks like software vulnerabilities are present. (\cite{Xiaohui6643029}; \cite{Zhao6746513}).
			\4 Data at the application layer must be protected through authentication. (\cite{Zhao6746513}).
			
\1 Evaluating Protocols for IoT Security
	\2 The Internet of Things needs a complete and standardized security stack. 
	\2 CoAP
		\3 CoAP suggestions for security standards
	\2 IEEE 802.15.4
	\2 6LoWPaN
	\2 Datagram Transport Layer Security (DTLS)
		\3 DTLS is TLS but over the UDP protocol.
		\3 Mandatory standard for protection of CoAP messages (\cite{Garcia-Morchon:2013:SII:2462096.2462117}).
		\3 DTLS was designed for traditional networks, but may be adopted for use with the IoT. (\cite{Keoh6817545}).
		\3 DTLS allows for good interoperability, but lacks in efficiency. (\cite{Garcia-Morchon:2013:SII:2462096.2462117}).
	\2 Host Identity Protocol (HIP)
		\3 HIP works by using Diffie-Hellman key exchange. 
		\3 The literature found that HIP is more efficient than DTLS. (\cite{Garcia-Morchon:2013:SII:2462096.2462117}).
			\4 HIP has less communication overhead.
			\4 HIP also has less of a memory footprint.
		\3 HIP Diet Exchange (HIP DEX)
			\4 Extension to HIP that uses lightweight and static Diffie-Hellman authentication. (\cite{Hummen}).
			\4 HIP DEX is optimized for low power devices. 
			\4 Some literature has proposed extensions to HIP DEX which improve efficiency. (\cite{Hummen}).
\1 Government Standards for the IoT  
	\2 Current legal environment
		\3 IoT devices do not currently have much regulation.
		\3 Industry is self-regulating. (\cite{Weber201023}). 
	\2 Requirements for legislation
		\3 Literature recommends that countries enact similar legislation due to the amount of companies that sell products on a 
			global scale. (\cite{Weber201023}). 
	\2 U.S. Government regulation
		\3 Federal Trade Commission Report (\cite{ftc}).
			\4 FTC recommends companies design products with security in mind from the start.
			\4 FTC recommends companies minimize data collected on users.
			\4 The FTC did not recommend any new legislation; however, they did recommend that Congress enact general data security legislation.
		\3 House Energy and Commerce Committee meeting about IoT security
		\3 Senate Commerce Committee meeting about IoT Security
	\2 EU regulation
		\3 EU lawmakers propose labeling requirement for IoT manufacturers (\cite{Stupp}).
		\3 EU also considering adopting new privacy requirements. (\cite{Kozlov}).

\1 Conclusion
\end{outline}
\end{document}
