\documentclass{article}
\usepackage[american]{babel}
\usepackage[utf8]{inputenc}
\usepackage[margin=1in]{geometry}
\usepackage[citestyle=apa,style=apa,backend=biber]{biblatex}
\DeclareLanguageMapping{american}{american-apa}
\addbibresource{bibliography.bib}
\usepackage{outlines}
\usepackage{enumitem}
\setenumerate[1]{label=\Roman*.}
\setenumerate[2]{label=\Alph*.}
\setenumerate[3]{label=\roman*.}
\setenumerate[4]{label=\alph*.}

\begin{document}
\begin{outline}[enumerate]
\1 Introduction
	\2 Internet of Things (IoT) security is currently not adequate.
		\3 Many devices are vulnerable to various types of attacks.
		\3 In September 2016, a security researcher had his website attacked by thousands of compromised IoT devices. (\cite{Krebs}; \cite{Incapsula}).
	\2 Overview of the Literature
		\3 The literature outlines security risks in all layers of the IoT architecture. (\cite{Xiaohui6643029}; \cite{Zhao6746513}; \cite{Suo6188257}).
			\4 However, there is some debate as to which layers need the most attention for future research. (\cite{Zhao6746513}; \cite{Kozlov}).
		\3 The IoT needs a complete stack of protocols to ensure security. 
			\4 The Datagram Transport Layer Security (DTLS) protocol has received a lot of attention from researchers. 
			\4 However, other literature suggests that the Host Identity Protocol (HIP) would be a better choice. 
			
\1 Methods
\1 Security Risks for IoT Devices
\1 Evaluating Protocols for IoT Security
	\2 Datagram Transport Layer Security (DTLS)
	\2 Host Identity Protocol (HIP)
		\3 Comparison with DTLS
		\3 HIP Diet Exchange (HIP DEX)

\1 Government Standards for the IoT  
	\2 Requirements for legislation
	\2 U.S. Government regulation
		\3 Federal Trade Commission Report
	\2 EU regulation

\1 Conclusion
\end{outline}
\end{document}
