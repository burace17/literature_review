\documentclass{article}
\usepackage[american]{babel}
\usepackage[citestyle=apa,style=apa,backend=biber]{biblatex}
\usepackage{scrextend}
\usepackage[margin=1in]{geometry}
\DeclareLanguageMapping{american}{american-apa}
\addbibresource{bibliography.bib}

\newenvironment{annotation}{\begin{addmargin}[2.5em]{0em} \begin{flushleft}}{\end{flushleft} \end{addmargin}}

\begin{document}
\fullcitebib{Zhao6746513}
\begin{annotation}
The authors present a review of what the current literature says about Internet of Things Security. 
It details the security problems in the perception, network, and application layers of the Internet of Things structure. 
It also presents the current security mechanisms present in these layers. 
This source will help me answer my research questions regarding the current security risks and the current security mechanisms present in the Internet of Things currently.
This article also provides a useful flowchart to allow readers to visualize the security mechanisms present in the Internet of Things.  
\end{annotation}

\fullcitebib{Granjal7005393}
\begin{annotation}
This article presents an analysis of the state of security in the Internet of Things (IoT). It defines the four layers of the IoT architecture and presents the security
problems present in each layer. It also discusses the current research regarding IoT security. Currently, there is a lot of research being done on how to utilize existing
encryption algorithms on low power and speed devices that are prominant in the IoT. I plan to use this article to answer my questions regarding the current state of IoT
security and what is being done to mitigate the existing security risks. 
\end{annotation}

\fullcitebib{Kozlov}
\begin{annotation}
This article analyzes the architecture of the IoT. It breaks the architecture down into six layers, unlike other literature which breaks it down into three. The extra layers may provide a more precise overview of the security problems in each layer. This article also gives real world examples of security threats faced by IoT devices. Finally, the article provides information about pending European Union legislation regarding the IoT. The legislation mainly involves user privacy rights. I plan to use this article to find examples of security threats and to provide some information about pending legislation in Europe. 
\end{annotation}

\fullcitebib{Suo6188257}
\begin{annotation}

\end{annotation}
\end{document}

