\documentclass{article}
\usepackage[american]{babel}
\usepackage[citestyle=apa,style=apa,backend=biber]{biblatex}
\usepackage{scrextend}
\usepackage[margin=1in]{geometry}
\DeclareLanguageMapping{american}{american-apa}
\addbibresource{bibliography.bib}

\newenvironment{annotation}{\begin{addmargin}[2.5em]{0em} \begin{flushleft}}{\end{flushleft} \end{addmargin}}

\begin{document}
\fullcitebib{Zhao6746513}
\begin{annotation}
The authors present a review of what the current literature says about Internet of Things Security. 
It details the security problems in the perception, network, and application layers of the Internet of Things structure. 
It also presents the current security mechanisms present in these layers. 
This source will help me answer my research questions regarding the current security risks and the current security mechanisms present in the Internet of Things currently.
This article also provides a useful flowchart to allow readers to visualize the security mechanisms present in the Internet of Things.  
\end{annotation}

\fullcitebib{Granjal7005393}
\begin{annotation}
This article presents an analysis of the state of security in the Internet of Things (IoT). It defines the four layers of the IoT architecture and presents the security
problems present in each layer. It also discusses the current research regarding IoT security. Currently, there is a lot of research being done on how to utilize existing
encryption algorithms on low power and speed devices that are prominant in the IoT. I plan to use this article to answer my questions regarding the current state of IoT
security and what is being done to mitigate the existing security risks. 
\end{annotation}

\fullcitebib{Kozlov}
\begin{annotation}
This article analyzes the architecture of the IoT. It breaks the architecture down into six layers, unlike other literature which breaks it down into three. The extra layers may provide a more precise overview of the security problems in each layer. This article also gives real world examples of security threats faced by IoT devices. Finally, the article provides information about pending European Union legislation regarding the IoT. The legislation mainly involves user privacy rights. I plan to use this article to find examples of security threats and to provide some information about pending legislation in Europe. 
\end{annotation}

\fullcitebib{Suo6188257}
\begin{annotation}
The authors provide an overview of the security requirements at each level of the IoT architecture. They also discuss the current status of the research on various security mechanisms currently used
in the IoT. They outline areas where not a lot of research has been done and what the previous research has focused on. Finally, the authors discuss what future challenges the literature states
IoT devices will face in the future. I plan to use this article to give some background information on what kind of research has been done in the past and where the gaps in that research are. 
\end{annotation}

\fullcitebib{Roman6017172}
\begin{annotation}
Many of the traditional protocols used to secure services on the Internet cannot be used with the Internet of Things due to the low power devices commonly used. This article provides examples of protocols
currently used with IoT devices and others that are being considered for future use. It also provides examples of government recommendations for security mechanisms. The article also identifies a gap in the 
research regarding the negotiation of dynamic keys between unknown entities. I will use this article to provide examples of standards and protocols currently in use and what is being considered for future use. I will also mention the gap in the literature identified by this article.
\end{annotation}

\fullcitebib{Roman20132266}
\begin{annotation}
There are several different ways to design networks that include the Internet of Things. They include the centralized apprach, collaborative apprach, and the distributed approach. This article focuses on the
distributed approach, but provides background information on the others. It provides examples and diagrams about how the networks are designed. It then goes into the security issues found in the distributed
approach specifically. Finally, the article provides an overview of common attacks used on Internet of Things devices. I plan to use this article to address my research question regarding current security
risks IoT devices face today. I also plan to use this article to provide background information on network designs that utilize the IoT. 
\end{annotation}

\fullcitebib{Keoh6817545}
\begin{annotation}
This article provides an overview on the protocol standardization efforts currently under way for the IoT. It focuses on the Datagram Transport Layer Security (DTLS) protocol, which has also been the focus of most
existing research. The authors give an in depth explanation of how the protocol works and analyze the performance of the protocol on low power devices. Finally, they outline what needs to be done in to 
ensure that DTLS can be used as the standard for providing security to IoT devices. I plan to use the article to provide information about existing standardization efforts and the strengths and weaknesses of DTLS. 
\end{annotation}

\fullcitebib{Garcia-Morchon:2013:SII:2462096.2462117}
\begin{annotation}
The Datagram Transport Layer Security (DTLS) protocol has received the most attention from researchers. However, there are other options for securing the IoT. The authors for this article performed an 
experiment comparing DTLS to various implementations of the Host Identity Protocol (HIP). The experiment consisted of securing the IoT in terms of network access, key management, and secure communication.
They found that HIP had a smaller memory footprint and had less communication overhead than DTLS. I plan to use this article to discuss an alternative standard that has been shown to be more efficient
than the commonly studied standard, DTLS.
\end{annotation}

\fullcitebib{ftc}
\begin{annotation}
In January 2015, the Federal Trade Commission (FTC) released a report detailing their recommendations to manufacturers regarding the Internet of Things security. They recommended that companies design their
products with security in mind from the start. They also suggested that companies minimize the data they connect on customers and notify customers if their data is being collected. With regard to 
legislation, the FTC did not recommend any new laws regulating the IoT. However, they recommended that Congress enact general data security legislation that will increase IoT security and technology 
security in general. I plan to use this report to answer my research question about government action on IoT security.
\end{annotation}

\newpage
\fullcitebib{Capossele}
\begin{annotation}
The Datagram Transport Layer Security (DTLS) protocol has been widely considered for use with the IoT, but one key problem has been its efficiency. The authors for this article present their
own solution for improving the efficiency of the protocol. Through the use of optimized assembly code, they were able to reduce the number of memory operations when doing arithmetic on large
integers. They were also able to reduce the execution time of signature verifications. I plan to use this article to demonstrate a solution to a widely acknowledged problem in the literature which
is the efficiency of the DTLS protocol. Other literature suggests abandoning the protocol and I will use this source to present another approach. 
\end{annotation}

\fullcitebib{Incapsula}
\begin{annotation}
Most literature cites theoretical examples of attacks that IoT devices can face but does not provide any real world examples of those attacks. This article analyzes a recent distributed denial of 
service attack on a prominent security researcher's website. The attacker used thousands of IoT devices that had been infected with malware to carry out the attack. The article also analyzes the
recently released source code for the malware and speculates about the origin of the malware. I will use this article to provide a real world example of an attack being committed using IoT devices with poor security. 
\end{annotation}

\fullcitebib{Stupp}
\begin{annotation}
This article reports on European Union lawmakers plans to draft new rules for IoT devices. One of the proposals would add a labeling system similar to home appliances that would
certify if a device meets security guidances. The rules would encourage, but not require, hardware manufacturers to get their devices certified. The rules would also regulatate how companies 
can use consumers' information. The article did not mention when the commission would consider the labeling provision. I will use this article to provide an example of what is currently being considered in Europe regarding the regulation of the Internet of Things. 
\end{annotation}

\fullcitebib{Hummen}
\begin{annotation}
This article proposes several extensions to the Host Identity Protocol Diet EXchange (HIP DEX) which improve security and performance. The issues the article identifies include the performance impact of 
Diffie-Hellman operations, the denial of service attack risk during handshake, and the performance impact of public key operations during retransmission. The authors' extensions improve on HIP DEX in all of
the areas mentioned. This article will be used in the final report to provide information about improvements being considered for HIP DEX and how the standard compares to the Datagram Transport Layer Security protocol. 
\end{annotation}

\newpage
\fullcitebib{Xiaohui6643029}
\begin{annotation}
The author for this report analyzes the security risks IoT devices face today. The report mentions authentication as being a very important component for the IoT. Devices at the perception layer must be
properly authenticated with the rest of the network in order for the data received to be trusted. However, one major issue is that many IoT devices are not powerful enough for existing authentication methods.
The report also mentions various security mechanisms IoT devices can use. They include certification and encryption. I plan to use this article to provide information about current security risks
faced by IoT devices.
\end{annotation}

\fullcitebib{Zhang:2015:EST:2714576.2737091}
\begin{annotation}
This article provides four examples of security mechanisms being considered for use in the IoT. They include authentication by gateway, security token, trust chain, and global trust tree. Each of the 
approaches has trade-offs, however.
For example, a major drawback to authentication by gateway is that the network has a single point of failure. If the gateway is not working, the IoT devices will not be able to transmit their data.
 The article also provides information about current security threats and privacy issues. I will mainly use this article to provide examples of security mechanisms for
IoT devices. 
\end{annotation}

\fullcitebib{Hummen3}
\begin{annotation}
The authors for this report analyze ways to make the use of certificates for authentication with IoT devices viable. They identify existing problems such as certificate authentication exceeding the
frame size specified by IEEE 802.15.4. They also identify significant overhead from verifying the validity of the certificates, signature verification, and Diffie-Hellman key agreement. One of their
proposed solutions is to setup a gateway that verifies the certificate chain between the devices. This allows the expenisve processing to be done on a faster device. I plan to use this article to discuss
new security mechanisms that are being considered.
\end{annotation}

\fullcitebib{Khari}
\begin{annotation}

\end{annotation}
\end{document}

