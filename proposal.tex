\documentclass[letterpaper, 12pt]{texMemo}
\usepackage[american]{babel}
\usepackage[utf8]{inputenc}
\usepackage[citestyle=apa,style=apa,backend=biber]{biblatex}
\DeclareLanguageMapping{american}{american-apa}
\addbibresource{bibliography.bib}
\memoto{Dr. William Hsu}
\memofrom{Blair Urish}
\memosubject{Proposal to write a literature review examining the security of the Internet of Things}
\memodate{\today}

\newenvironment{annotation}{\begin{addmargin}[2.5em]{0em} \begin{flushleft}}{\end{flushleft} \end{addmargin}}

\begin{document}
\maketitle
\begin{flushleft}
\subsection*{Introduction}
The purpose of this memo is to propose a literature review for analyzing the security of the Internet of Things (IoT). 
The proposed literature review will analyze the current security risks and synthesize the research being done to develop new standards and protocols. The literature review will also provide
an overview of government action for improving IoT security.
This document will include some background information about the topic, the specific research questions that will be addressed, an outline of
the final report, and the methodology used to create the report.

\subsection*{Background}
This section will discuss background information related to the report's topic. The section will contain the historical background of the issue, a brief
overview of the current research, and discuss the purpose of this research. \\
~\newline
\textit{Historical Background}\\ 
~\newline
By 2020, it is expected that there will be 
25 million Internet of Things devices connected to the Internet. (\cite{Martinez1}). With so many devices, it is important that manufacturers take security very seriously. 
In September 2016, a security researcher named Brian Krebs had his website temporarily taken down due to a Distributed Denial of Service attack. The attacker
used thousands of malware-infected IoT devices and sent 600 Gigabits per second of traffic to Krebs' blog. (\cite{Krebs}). The malware, called Mirai, 
infected IoT devices with poor security. An analysis showed that the devices were mainly CCTV cameras, DVRs, and routers. (\cite{Incapsula}).\\
~\newline
A few months later, in December 2016, SEC Consult discovered a backdoor in Sony IPELA Engine IP Cameras that allowed for full remote access over the Internet. (\cite{sec}). The backdoor could allow an attacker
to install malicious code on the device. At the time of the disclosure, there were at least 4,250 devices at risk. (\cite{Krebs2}). Owners of the affected devices must manually update them
to be safe from potential attack. No malware has targeted these devices yet, but it is possible that may change in the future if the owners do not update their devices. Overall, these two incidents
are just a few examples of the problems with IoT security.\\

~\newline
\textit{Overview of Current Research}\\
~\newline
The literature outlines security risks in all layers of the Internet of Things architecture. (\cite{Xiaohui6643029}; \cite{Zhao6746513}; \cite{Suo6188257}). 
There is some debate as to which layers need the most attention for future research. Some argue that risks in the perception layer present the greatest risk. (\cite{Zhao6746513}).
Others argue that security at the perception layer is of a lower priority. (\cite{Kozlov}). \\
~\newline
As for security standards, the Constrained Application Protocol suggests the use of Datagram Transport Layer Security (DTLS). However, this introduces a large amount of overhead. (\cite{Capossele}).  
Other literature has considered the use of the Host Identity Protocol (HIP) instead. (\cite{Garcia-Morchon:2013:SII:2462096.2462117}). The literature found that HIP has less overhead when compared to
DTLS. Some literature has even proposed extensions to HIP that would further reduce overhead (\cite{Hummen}).\\
~\newline
\textit{Purpose of the Research}\\
~\newline
The purpose of this literature review is to synthesize the existing research regarding the security risks, standards, and government action into one report. There are many different points of view in the
literature about which areas of the IoT are most at risk. There is also conflict in the literature regarding which standards should be used and how they should be implemented. With this literature review,
all of this debate will be collected into one report which will make it easier for future research to be done. \\
~\newline

\subsection*{Proposal}
This section will give the specific research questions this literature review should answer and justify why the questions should be answered. This section will also provide a brief description of each section in the proposed literature review. \\
~\newline
\underline{Research Questions}\\
~\newline
\textit{What types of security risks do IoT devices face today?}\\
The purpose of this question is to investigate what types of threats there are to current IoT devices. In order to develop new standards and threat mitigation techniques,
the current threats must be understood. \\
~\newline
\textit{What types of security standards or protocols has the literature proposed and how will the standards be implemented?}\\
This question will address the current research being done in order to design and implement new security standards. If there is any conflict in the literature over this issue,
that will also be addressed. \\
~\newline
\textit{Are there any government standards in place or are there any being considered?}\\
Many governments provide standards for certain industries. This question will examine if there are any in place for the IoT or if there are any being
currently considered.\\ 
~\newline
\underline{Report Description}\\
~\newline
\textit{Introduction}\\
The section will provide background information about the state of Internet of Things security.\\ 
~\newline
\textit{Security Risks for Current IoT Devices}\\
This section will provide a comprehensive look into the current security risks IoT devices face today. \\
~\newline
\textit{Security Standards and Protocols}\\
This section will synthesize research being done to develop new security standards and protocols. If there is any debate found in
the literature, that will be explored as well.\\ 
~\newline
\textit{Current and Future Government Standards}\\
This section will provide an overview of current standards imposed by the United States government and the European Union. The section will also look at 
future standards being considered.\\
~\newline
\textit{Conclusion}\\
This section will summarize the findings of the report.\\
~\newline

\subsection*{Methods}
~\newline
To create the report, information from academic journals and conference proceedings will be used. These articles are found in 
databases such as Scopus, IEEE, and the ACM. Search terms such as "Internet of Things" and "security" will be used to locate
relevant articles in the databases. Web sources from industry leaders will be used to provide background information surrounding
the topic.\\ 
~\newline

\subsection*{Conclusion}
~\newline
The Internet of Things is a growing industry that must keep security in mind. The proposed literature review will gather relevant 
research about the current security risks and proposed standards for the IoT while pointing out any gaps or conflict in the literature along the way. 
The report will also gather information about current and future government standards.

\newpage
Bibliography\\
~\newline
\fullcite{Capossele}\\
~\newline
\fullcite{Garcia-Morchon:2013:SII:2462096.2462117}\\
~\newline
\fullcite{Incapsula}\\
~\newline
\fullcite{Hummen}\\
~\newline
\fullcite{Kozlov}\\
~\newline
\fullcite{Krebs2}\\
~\newline
\fullcite{Krebs}\\
~\newline
\fullcite{Martinez1}\\
~\newline
\fullcite{Roman6017172}\\
~\newline
\fullcite{sec}\\
~\newline
\fullcite{Suo6188257}\\
~\newline
\fullcite{Xiaohui6643029}\\
~\newline
\fullcite{Zhao6746513}\\
\end{flushleft}
\end{document}
