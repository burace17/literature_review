\documentclass[letterpaper, 12pt]{texMemo}
\usepackage[american]{babel}
\usepackage[utf8]{inputenc}
\usepackage[citestyle=apa,style=apa,backend=biber]{biblatex}
\DeclareLanguageMapping{american}{american-apa}
\addbibresource{bibliography.bib}
\memoto{Dr. William Hsu}
\memofrom{Blair Urish}
\memosubject{Proposal to write a literature review examining the security of the Internet of Things}
\memodate{\today}

\newenvironment{annotation}{\begin{addmargin}[2.5em]{0em} \begin{flushleft}}{\end{flushleft} \end{addmargin}}

\begin{document}
\maketitle
\begin{flushleft}
\subsection*{Introduction}
The purpose of this memo is to propose a literature review for analyzing the security of the Internet of Things (IoT). 
The literature review will analyze the current security risks, features, and standards for IoT devices. This memo will also provide
an overview of any future security standards that are being considered.
This document will include some background information about the topic, the specific research questions that will be addressed, an outline of
the final report, and the methodology used to create the report.

\subsection*{Background}
\textit{Historical Background}\\ 
~\newline
An Internet of Things device can be defined as a physical "thing" that is connected to the Internet. By 2020, it is expected that there will be 
25 million IoT devices connected to the Internet. (\cite{Martinez1}). With so many devices, it is important that manufacturers take security very seriously. 
In September 2016, a security researcher named Brian Krebs had his website temporarily taken down due to a Distributed Denial of Service attack. The attacker
used thousands of malware-infected IoT devices and sent 600 Gigabits per second of traffic to Krebs' blog. (\cite{Krebs}). The malware, called Mirai, 
infected IoT devices with poor security. An analysis showed that the devices were mainly CCTV cameras, DVRs, and routers. (\cite{Incapsula}).\\
~\newline
\textit{Overview of Current Research}\\
~\newline
The literature outlines security risks in all layers of the Internet of Things architecture. (\cite{Xiaohui6643029}) (\cite{Zhao6746513}) (\cite{Suo6188257}). 
There is some debate as to which layers need the most attention for future research. Some argue that risks in the perception layer present the greatest risk. (\cite{Zhao6746513}).
Others argue that security at the perception layer is of a lower priority. (\cite{Kozlov}). As for the current security features, the network layer has the most
complete feature set due to lots of previous research. (\cite{Suo6188257}). Organizations such as ISO have standards in place for certain IoT devices, but manufacturers
are not required to follow them. There are also government standards in development by the EU. (\cite{Roman6017172}) \\
~\newline
\textit{Purpose of the Research}
~\newline
As more IoT devices are connected to the Internet, the threat of further attacks due to a lack of security becomes even higher. The damage these attacks could
cause will also become higher. The purpose of this literature review is to synthesize the current research on this topic into one article and to find areas that require 
further research.\\
~\newline

\subsection*{Proposal}
\underline{Research Questions}\\
~\newline
\textit{What types of security risks do IoT devices face today?}\\
The purpose of this question is to investigate what types of threats there are to current IoT devices. In order to develop new standards and threat mitigation techniques,
the current threats must be understood. \\
~\newline
\textit{What types of security features do IoT devices have now and are they effective against current threats?}\\
In order for new security features to be developed, one must fully understand the current features and how effective they have been. \\
~\newline
\textit{What new security features have been proposed?}\\
This question will address the current research being done on developing new security features for IoT devices. It will also provide a review of any conflict that exists
in the literature regarding this issue. \\
~\newline
\textit{Are there any security standards in place or are being considered?}\\
Many organizations and governments provide standards for certain industries. This question will examine if there are any in place for the IoT or if there are any being
currently considered.\\ 
~\newline
\underline{Report Description}\\
~\newline
\textit{Introduction}\\
The section will provide background information about the state of Internet of Things security.\\ 
~\newline
\textit{Security Risks for Current IoT Devices}\\
This section will provide a comprehensive look into the current security risks IoT devices face today. \\
~\newline
\textit{Security Features of Current IoT Devices}\\
This section will analyze what security features IoT devices have today and how effective they are.\\
~\newline
\textit{Future Security Features}\\
This section will synthesize research being done to develop new security features. It will also mention any conflict that 
exists in the literature.\\
~\newline
\textit{Current and Future Security Standards}\\
This section will provide an overview of current standards imposed by organizations and governments. It will also look at 
future standards being considered.\\
~\newline
\textit{Conclusion}\\
This section will summarize the findings of the report.\\
~\newline

\subsection*{Methods}
~\newline
To create the report, information from academic journals and conference procedings will be used. These articles are found in 
databases such as Scopus, IEEE, and the ACM. Search terms such as "Internet of Things" and "security" will be used to locate
relavent articles in the databases. Web sources from industry leaders will be used to provide background information surrounding
the topic.\\ 
~\newline

\subsection*{Conclusion}
~\newline
The Internet of Things is a growing industry that must keep security in mind. The proposed literature review will gather relevant 
research regarding the current security risks, features, and standards in the IoT. It will also point out any gaps or conflict in
the literature. 

\newpage
Bibliography\\
~\newline
\fullcite{Incapsula}\\
~\newline
\fullcite{Kozlov}\\
~\newline
\fullcite{Krebs}\\
~\newline
\fullcite{Martinez1}\\
~\newline
\fullcite{Roman6017172}\\
~\newline
\fullcite{Suo6188257}\\
~\newline
\fullcite{Xiaohui6643029}\\
~\newline
\fullcite{Zhao6746513}\\
\end{flushleft}
\end{document}
