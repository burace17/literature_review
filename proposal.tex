\documentclass[letterpaper, 12pt]{texMemo}
\usepackage[american]{babel}
\usepackage[citestyle=apa,style=apa,backend=biber]{biblatex}
\DeclareLanguageMapping{american}{american-apa}
\addbibresource{bibliography.bib}
\memoto{Dr. William Hsu}
\memofrom{Blair Urish}
\memosubject{Proposal to write a literature review examining the security of the Internet of Things}
\memodate{February 27, 2017}

\newenvironment{annotation}{\begin{addmargin}[2.5em]{0em} \begin{flushleft}}{\end{flushleft} \end{addmargin}}

\begin{document}
\maketitle
\begin{flushleft}
\subsection*{Introduction}
The purpose of this memo is to propose a literature review for analyzing the security of the Internet of Things (IoT). 
This literature review will first synthesize the current research regarding the security risks IoT devices face today. 
Next, it will synthesize the research regarding current security features and risk mitigation techniques.  
Finally, the literature review will analyze the current standards for IoT devices and any future standards that are being considered.  

\subsection*{Background}
\textit{Definition of the Internet of Things}\\ 
~\newline
  The Internet of Things refers to the connection of physical "things" to the Internet. (\cite{Kozlov}). They are commonly referred to as "smart" devices. 
Through the IoT, the Internet has found its way into many household items, including refrigerators, TVs, door locks, toasters, and many others. 
However, introducing the Internet to all of these new areas has also introduced security risks. Computers have always faced security issues, and now household items
face similiar risks.\\ 
~\newline
\textit{Security Risks}
~\newline


\newpage
Bibliography\\
~\newline
  \fullcite{Kozlov}
\end{flushleft}
\end{document}
