\documentclass[letterpaper, 12pt]{texMemo}
\usepackage[american]{babel}
\usepackage[citestyle=apa,style=apa,backend=biber]{biblatex}
\DeclareLanguageMapping{american}{american-apa}
\addbibresource{bibliography.bib}
\memoto{Dr. William Hsu}
\memofrom{Blair Urish}
\memosubject{Proposal to write a literature review examining the security of the Internet of Things}
\memodate{February 27, 2017}

\newenvironment{annotation}{\begin{addmargin}[2.5em]{0em} \begin{flushleft}}{\end{flushleft} \end{addmargin}}

\begin{document}
\maketitle
\begin{flushleft}
\subsection*{Introduction}
The purpose of this memo is to propose a literature review for analyzing the security of the Internet of Things (IoT). 
This literature review will first synthesize the current research regarding the security risks IoT devices face today. 
Next, it will synthesize the research regarding current security features and risk mitigation techniques.  
Finally, the literature review will analyze the current standards for IoT devices and any future standards that are being considered.  

\subsection*{Background}
\textit{Definition of the Internet of Things}\\ 
~\newline
  The Internet of Things refers to the connection of physical "things" to the Internet. (\cite{Kozlov}). They are commonly referred to as "smart" devices. 
Through the IoT, the Internet has found its way into many household items, including refrigerators, TVs, door locks, toasters, and many others. 
However, introducing the Internet to all of these new areas has also introduced security risks. Computers have always faced security issues, and now household items
face similiar risks.\\ 
~\newline

\textit{Security Risks}\\
~\newline
The Internet of Things does not just include the devices themselves. It also includes the applications used to control them and the network infrastructure used for connections. 
This architecture is generally separated into three layers: perception, network, and application. (\cite{Zhao6746513}). The perception layer includes the physical devices themselves. 
The application layer would include applications used to control the devices in the perception layer. Some literature also defines a support layer, which helps connect the perception layer
to the network layer. (\cite{Suo6188257}). Finally, the literature defines the network layer. It allows the perception layer to transfer information to the application layer.\\ 
~\newline

\textit{Security Risks in the Perception Layer}\\
~\newline
In the perception layer, physical security is a major concern. Many devices are in public areas where their data could be easily intercepted. Some literature states that the interception of
data at this layer could pose a security risk to the entire network. (\cite{Zhao6746513}). However, other literature says the need for security in the perception layer is of lower priority and that
the integrity of the data should be verified at a different layer. (\cite{Kozlov}).\\
~\newline

\textit{Security Risks in the Network Layer}\\
~\newline
More is known about security risks in the network layer because these risks are not unique to the IoT. However, attacks such as man-in-the-middle and distributed denial of service are still common. 
Literature considers the security in the network layer to be fairly complete. (\cite{Kozlov}).\\
~\newline

\textit{Security Risks in the Application Layer}\\
~\newline


\newpage
Bibliography\\
~\newline
\fullcite{Kozlov}\\
~\newline
\fullcite{Zhao6746513}\\
~\newline
\fullcite{Suo6188257}\\
\end{flushleft}
\end{document}
