\documentclass[letterpaper, 12pt]{article}
\usepackage[american]{babel}
\usepackage[utf8]{inputenc}
\usepackage[citestyle=apa,style=apa,backend=biber]{biblatex}
\usepackage[margin=1in]{geometry}
\usepackage{graphicx}
\usepackage{caption}
\usepackage{float}
\setlength\bibitemsep{2\itemsep}
\DeclareLanguageMapping{american}{american-apa}
\addbibresource{bibliography.bib}

\pagenumbering{roman}

\begin{document}
\begin{titlepage}
\centering
	\vspace*{5.75cm}
	{\huge\bfseries A literature review of Internet of Things security\par}
	\vspace{2cm}
	Blair Urish\\
	Kansas State University\\
	College of Engineering\\
	Department of Computer Science\\
	\vspace{1cm}
	Dr. William Hsu\\
	Professor\\
	Department of Computer Science\\
	\vspace{1cm}
	May 2017
\end{titlepage}


\begin{abstract}
\thispagestyle{plain}
\setcounter{page}{2}
\begin{flushleft}
	Lorem ipsum dolor sit amet, consectetur adipiscing elit, sed do 
eiusmod tempor incididunt ut labore et dolore magna aliqua. Ut 
enim ad minim veniam, quis nostrud exercitation ullamco. 
Lorem ipsum dolor sit amet, consectetur adipiscing elit, sed do 
eiusmod tempor incididunt ut labore et dolore magna aliqua. Ut 
enim ad minim veniam, quis nostrud exercitation ullamco laboris
Lorem ipsum dolor sit amet, consectetur adipiscing elit, sed do 
eiusmod tempor incididunt ut labore et dolore magna aliqua. Ut 
enim ad minim veniam, quis nostrud exercitation ullamco laboris
\end{flushleft}
\end{abstract}

\newpage
\setcounter{page}{3}
\tableofcontents
\newpage

\cleardoublepage
\addcontentsline{toc}{section}{\listfigurename}
\listoffigures
\newpage

\pagenumbering{arabic}
\begin{flushleft}
\section*{Introduction}
\addcontentsline{toc}{section}{Introduction}
This section will discuss background information related to the report's topic. The section will contain the historical background of the issue, a brief
overview of the current research, and discuss the purpose of this research. \\
~\newline
\textit{Historical Background}\\ 
~\newline
By 2020, it is expected that there will be 
25 million Internet of Things devices connected to the Internet. (\cite{Martinez1}). With so many devices, it is important that manufacturers take security very seriously. 
In September 2016, a security researcher named Brian Krebs had his website temporarily taken down due to a Distributed Denial of Service attack. The attacker
used thousands of malware-infected IoT devices and sent 600 Gigabits per second of traffic to Krebs' blog. (\cite{Krebs}). The malware, called Mirai, 
infected IoT devices with poor security. An analysis showed that the devices were mainly CCTV cameras, DVRs, and routers. (\cite{Incapsula}).\\
~\newline
A few months later, in December 2016, SEC Consult discovered a backdoor in Sony IPELA Engine IP Cameras that allowed for full remote access over the Internet. (\cite{sec}). The backdoor could allow an attacker
to install malicious code on the device. At the time of the disclosure, there were at least 4,250 devices at risk. (\cite{Krebs2}). Owners of the affected devices must manually update them
to be safe from potential attack. No malware has targeted these devices yet, but it is possible that may change in the future if the owners do not update their devices. Overall, these two incidents
are just a few examples of the problems with IoT security.\\
~\newline
\textit{Overview of Current Research}\\
~\newline
The literature outlines security risks in all layers of the Internet of Things architecture. (\cite{Xiaohui6643029}; \cite{Zhao6746513}; \cite{Suo6188257}). 
There is some debate as to which layers need the most attention for future research. Some argue that risks in the perception layer present the greatest risk. (\cite{Zhao6746513}).
Others argue that security at the perception layer is of a lower priority. (\cite{Kozlov}). \\
~\newline
As for security standards, the Constrained Application Protocol suggests the use of Datagram Transport Layer Security (DTLS). However, this introduces a large amount of overhead. (\cite{Capossele}).  
Other literature has considered the use of the Host Identity Protocol (HIP) instead. (\cite{Garcia-Morchon:2013:SII:2462096.2462117}). The literature found that HIP has less overhead when compared to
DTLS. Some literature has even proposed extensions to HIP that would further reduce overhead (\cite{Hummen}).\\
~\newline
\textit{Purpose of the Research}\\
~\newline
The purpose of this literature review is to synthesize the existing research regarding the security risks, standards, and government action into one report. There are many different points of view in the
literature about which areas of the IoT are most at risk. There is also conflict in the literature regarding which standards should be used and how they should be implemented. With this literature review,
all of this debate will be collected into one report which will make it easier for future research to be done. \\
~\newline
\textit{Structure of Report}\\
~\newline
The report will first cover the methods used to gather the relevant research. Next, the report will discuss the security risks for current IoT devices. Then, the report will synthesize the research being
done to develop new security standards and protocols. The report will then discuss current and future government standards for IoT security. Finally, the findings of the report will be summarized in the conclusion
section.

\section*{Methodology}
\addcontentsline{toc}{section}{Methodology}

To create the report, information from academic journals and conference proceedings were used. These articles were found in 
databases such as Scopus, IEEE, and the ACM. Search terms such as "Internet of Things" and "security" were used to locate
relevant articles in the databases. Web sources from industry leaders were used to provide background information surrounding
the topic.\\ 


\section*{Security Risks for IoT Devices}
\addcontentsline{toc}{section}{Security Risks for IoT Devices}

This section will discuss the security risks faced by current IoT devices. The section will use the three main layers of the IoT architecture to break down the risks. Before the risks at each layer are discussed,
the IoT architecture will be formally defined. After that the risks found in the perception layer, network layer, and application layer will be discussed.

\subsection*{Defining the IoT Architecture}
\addcontentsline{toc}{subsection}{Defining the IoT Architecture}
The literature generally breaks the IoT architecture down into three layers: perception, network, and application. (\cite{Zhao6746513}; \cite{Xiaohui6643029}). Some literature adds additional layers to further narrow down the purpose of each
layer. (\cite{Granjal7005393}; \cite{Kozlov}; \cite{Suo6188257}). However, for the purposes of this report, three layers will be sufficient. The perception layer 
generally includes the devices that allow an IoT device to interact with the physical world. A temperature sensor would be an example of a
perception layer device. The network layer includes devices that allow the perception layer and application layer to interact through the Internet.
Traditional network hardware is included in this layer. Finally, the application layer is defined as the software that allows end users to control
and retrieve data from the perception layer devices. Below is a figure that gives more examples of devices that are commonly found at each layer.

\begin{figure}[H]
	\includegraphics[width=\linewidth,height=10cm,keepaspectratio]{figure2_new.png}
	\caption{Internet of Things Architecture (adapted from \cite{Zhao6746513})}
	\label{fig:arch}
\end{figure}

\subsection*{Security Risks in the Perception Layer}
\addcontentsline{toc}{subsection}{Security Risks in the Perception Layer}
In the perception layer, devices are commonly in public with minimal security. The literature states that there is a risk of an attacker physically
modifying a device at this layer. (\cite{Zhao6746513}; \cite{Xiaohui6643029}; \cite{Suo6188257}). Devices in the perception layer, for the most part,
are not very powerful. Due to this limitation, they are often unable to use the same security mechanisms found in traditional computers. 
(\cite{Suo6188257}; \cite{Granjal7005393}; \cite{Xiaohui6643029}). 

\newpage
\addcontentsline{toc}{section}{References}
\printbibliography
\end{flushleft}
\end{document}
